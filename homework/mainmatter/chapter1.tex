\chapter{Morden Control. Exerscise}

现代控制理论,高立群,清华大学出版社,第2版 
\vspace{2em}

\begin{question}[p60,2.1]
\end{question}
\begin{mdframed}[linewidth=0pt, backgroundcolor=gray!12]
    \begin{pf}
        令$C_1,C_2$两端电压为$u_{c_1},u_{c_2}$为状态变量$x_1,x_2$
        \begin{equation}
            u_{c_1}+C_1\frac{d u_{c_1}}{dt}\cdot R_2+u_{c_2}=u_1
        \end{equation}
        \begin{equation}
            C_1\frac{d u_{c_1}}{dt}+\frac{u_{c_1}}{R_1}=C_2\frac{du_{c_2}}{dt}
        \end{equation}

        整理
        \begin{equation}
            \frac{d u_{c_1}}{dt} = -\frac{R_1+R_2C_1}{R_2C_2}u_{c_1}+\frac{1}{R_2C_1}u_{c_2}+\frac{1}{R_2C_1}u_1
        \end{equation}
        \begin{equation}
            \frac{d u_{c_2}}{dt} = -\frac{1}{R_2C_2}u_{c_1}-\frac{1}{R_2C_2}u_{c_2}+\frac{1}{R_2C_2}u_1
        \end{equation}

        写成状态方程的形式
        \begin{equation}
            \left[
                \begin{array}{c}
                    \dot{x}_1\\
                    \dot{x}_2
                \end{array}
            \right]=
            \left[
                \begin{array}{cc}
                    -\frac{R_1+R_2C_1}{R_2C_2} & \frac{1}{R_2C_1} \\
                    -\frac{1}{R_2C_2}          & -\frac{1}{R_2C_2}
                \end{array}
            \right]
            \left[
                \begin{array}{c}
                    x_1\\
                    x_2
                \end{array}
            \right]+
            \left[
                \begin{array}{c}
                    \frac{1}{R_2C_1}\\
                    \frac{1}{R_2C_2}
                \end{array}
            \right]u_1
        \end{equation}

        输入与输出的关系
        \begin{equation}
            y=u_1-u_{c_1}=[-1,0]
            \left[
                \begin{array}{c}
                    x_1 \\
                    x_2 
                \end{array}
            \right]+
            u_1
        \end{equation}
    \end{pf}
\end{mdframed}

